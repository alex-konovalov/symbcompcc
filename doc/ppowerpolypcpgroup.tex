%%%%%%%%%%%%%%%%%%%%%%%%%%%%%%%%%%%%%%%%%%%%%%%%%%%%%%%%%%%%%%%%%%%%%%%%%
%%
%W ppowerpolypcpgroup.tex  GAP documentation     Dörte Feichtenschlager
%%

%%%%%%%%%%%%%%%%%%%%%%%%%%%%%%%%%%%%%%%%%%%%%%%%%%%%%%%%%%%%%%%%%%%%%%%%%
\Chapter{p-power-poly-pcp-groups}

Eick and Leedham-Green \cite{ELG08} defined for a prime <p> and a fixed
coclass <r> infinite coclass sequences. These sequences consist of finite 
<p>-groups of coclass <r>. For each infinite coclass sequence there exists a 
consistent pp-presentation (see 
Section~"Background on (polycyclic) parametrised presentations") 
such that if we choose a natural number for the parameter and possibly reduce
the exponents modulo the relative orders, we obtain a consistent polycyclic 
presentation for a group in the sequence; and for each group in the sequence 
there exists a natural number such that using this as a value for the 
parameter, we obtain a polycyclic presentation for the group.

We use these consistent pp-presentations to compute parametrised 
groups, which we call <p>-power-poly-pcp-groups. Furthermore, methods for 
these are presented. Without specifying the parameter we compute certain 
properties and using the <p>-power-poly-pcp-groups we do this for all groups 
they represent at once.

The <p>-power-poly-pcp-groups have a consistent pp-presentation with 
generators $g_1, \ldots, g_n, t_1, \ldots t_d$ and $c_1, \ldots, c_m$, for some 
non-negative integers <n>, <d> and <m>, and relations of the form, where 
$rel[i,j]$ stores the right hand sides of the relations (see 
Section~"Background on (polycyclic) parametrised presentations" for more 
information on pp-presentations),

%display{nontext}
$$
\eqalign{
&\, g_i^p=rel[i,i],\cr
&\, t_i^{expo} = rel[n+i,n+i],\cr
&\, c_i^{expo\_vec[i]} = rel[n+d+i,n+d+i],\cr
&\, g_i^{g_j} = rel[j,i], \cr
&\, t_i^{g_j} = rel[j,n+i], \cr
&\, t_i^{t_j} = rel[n+j,n+i],
}
$$
%display{text}
% g_i^p = rel[i,i],
% t_i^{expo} = rel[n+i,n+i],
% c_i^{expo\_vec[i]} = rel[n+d+i,n+d+i],
% g_i^{g_j} = rel[j,i], 
% t_i^{g_j} = rel[j,n+i],
% t_i^{t_j} = rel[n+j,n+i],
%enddisplay
where the $t_i$'s commute modulo $\langle c_1,\ldots, c_m\rangle$ and the 
$c_i$'s are central. So <rel> (see Section~"Obtaining p-power-poly-pcp-groups")
are the right hand sides of the relations, where some depend on the parameter. 
The relative orders <expo> and <expo\_vec[i]> of the generators $t_j$ and 
$c_i$ depend on the parameter.

%%%%%%%%%%%%%%%%%%%%%%%%%%%%%%%%%%%%%%%%%%%%%%%%%%%%%%%%%%%%%%%%%%%%%%%%%
\Section{Example}

In this section we present the well-known example of quaternion groups 
$Q_{2^{x+3}}$. They have a pp-presentation of the following form:

%display{nontext}
$$
\eqalign{
\{ g_1,g_2,t_1 \mid &g_1^{2} = t_1^{2^x},\, g_2^{g_1} = g_2
t_1^{-1+2^{x+1}},\cr
&\, g_2^{2} = t_1,\, t_1^{g_1} = t_1^{-1+2^{x+1}},\cr
&\, t_1^{2^{x+1}} = 1 \}.
}
$$
%display{text}
% { g_1,g_2,t_1|g_1^2 = t_1^{2^x}, g_2^{g_1} = g_2t_1^{-1+2^{x+1}},
%                       g_2^{2} = t_1, t_1^{g_1} = t_1^{-1+2^{x+1}}, 
%                       t_1^{2^{x+1}} = 1 }.
%enddisplay

%%%%%%%%%%%%%%%%%%%%%%%%%%%%%%%%%%%%%%%%%%%%%%%%%%%%%%%%%%%%%%%%%%%%%%%%%
\Section{Obtaining p-power-poly-pcp-groups}

To obtain <p>-power-poly-pcp-groups:

\>PPPPcpGroups( <rel>, <n>, <d>, <m>, <expo>, <expo_vec>, <prime>, <cc>, <name> ) F
\>PPPPcpGroups( <rec> ) F

returns the p-power-poly-pcp-groups described by the consistent 
pp-presentation with generators $g_1, \ldots, g_n$, $t_1, \ldots t_d$, 
$c_1, \ldots, c_m$, for some non-negative integers <n>, <d> and <m>, and 
relations of the form 

%display{nontext}
$$
\eqalign{
&\, g_i^p=rel[i,i],\cr
&\, t_i^{expo} = rel[n+i,n+i],\cr
&\, c_i^{expo\_vec[i]} = rel[n+d+i,n+d+i],\cr
&\, g_i^{g_j} = rel[j,i], \cr
&\, t_i^{g_j} = rel[j,n+i], \cr
&\, t_i^{t_j} = rel[n+j,n+i].
}
$$
%display{text}
% g_i^p = rel[i,i],
% t_i^{expo} = rel[n+i,n+i],
% c_i^{expo\_vec[i]} = rel[n+d+i,n+d+i],
% g_i^{g_j} = rel[j,i], 
% t_i^{g_j} = rel[j,n+i],
% t_i^{t_j} = rel[n+j,n+i].
%enddisplay

The input consists of the following:

%display{nonhtml}
\beginitems
`<rel>' & is the list of the right hand sides of the relations, where each 
relation is presented by a list consisting of tuples; the first entry <i> of 
a tuple is the index of the generator (if $i \le n$, then it represents 
generator $g_i$, if $n \< i \le d$, then it represents generator $t_{i-n}$ 
and otherwise it represents generator $c_{i-n-d}$) and the second entry of 
the tuple is the corresponding exponent. 
Note that the exponents of the $g_i$'s are saved as integers and all other 
exponents as lists, representing elements depending on the parameter. 

`<n>' & is the number of generators $g_i$, 

`<d>' & is the number of generators $t_i$, 

`<m>' & is the number of generators $c_i$, 

`<expo>' & is the relative order of all generators $t_i$; note that <expo> is
a list that represents an element depending on the parameter,

`<expo_vec>' & is the list of relative orders, where the <i>th entry of the 
list gives the relative order of the generator $c_i$; note that each 
relative order is a list that represents an element depending on the 
parameter, 

`<prime>' & is the underlying prime <p>,

`<cc>' & if the <p>-power-poly-pcp-groups represent an infinite coclass
sequence of <p>-groups of coclass <r>, then <cc> = <r>. If they represent
Schur extensions of groups in an infinite coclass sequence, then <cc> is
the coclass of the groups in this infinite coclass sequence.

`<name>' & a string to name the <p>-power-poly-pcp-groups. 

`<rec>' & is a record of the form
<rec( rel, expo, n, d, m, prime, cc, expo_vec, name )>. 
\enditems
%display{nontext}
%\beginitems
%`<rel>' & is the list of relations, where each relation is presented by a 
%list consisting of tuples; the first entry <i> of a tuple is the index of the 
%generator (if $i \le n$, then it represents generator $g_i$, if $n \< i \le d$,
%then it represents generator $t_{i-n}$ and otherwise it represents generator
%$c_{i-n-d}$) and the second entry of the tuple is the corresponding exponent. 
%Note that the exponents of the $g_i$'s are saved as integers and all other 
%exponents as lists, representing elements depending on the parameter. 
%`<n>' & is the number of generators $g_i$, 
%`<d>' & is the number of generators $t_i$, 
%`<m>' & is the number of generators $c_i$,
%`<expo>' & is the relative order of all generators $t_i$; note that expo is
%given as a list to represent an element depending on the parameter,
%`<expo_vec>' & is the list of relative orders, where the <i>th entry of the 
%list gives the relative order of the generator $c_i$; note that  each 
%relative order is given as a list to represent an element depending on the 
%parameter, 
%`<prime>' & is the underlying prime <p>,
%`<cc>' & if the <p>-power-poly-pcp-groups represent an infinite coclass
%sequence of <p>-groups of coclass <r>, then <cc> = <r>. If they represent
%Schur extensions of groups in an infinite coclass sequence, then <cc> is
%the coclass of the groups in this infinite coclass sequence.
%`<name>' & a string to name the <p>-power-poly-pcp-groups. 
%`<rec>' & is a record of the form
%<rec( rel, expo, n, d, m, prime, cc, expo_vec, name )>. 
%\enditems
%enddisplay

The pp-presentation is described at the beginning of Chapter 
"p-power-poly-pcp-group". Note that the consistency of the presentation is 
checked and that the presentation has to be consistent.

\beginexample
gap> ParPresGlobalVar_2_1[1];
rec( 
  rel := [ [ [ [ 1, 0 ] ] ], [ [ [ 2, 1 ], [ 3, -1+2*2^x ] ], [ [ 3, 1 ] ] ], 
      [ [ [ 3, -1+2*2^x ] ], [ [ 3, 1 ] ], [ [ 3, 0 ] ] ] ], expo := 2*2^x, 
  n := 2, d := 1, m := 0, prime := 2, cc := 1, expo_vec := [  ], name := "D" )
gap> G := PPPPcpGroups( ParPresGlobalVar_2_1[1] );
< P-Power-Poly-pcp-groups with 3 generators of relative orders [ 2,2,2*2^x ] >
\endexample

\>PPPPcpGroupsElement( <G>, <word> ) F

constructs an element in <p>-power-poly-pcp-groups, where <G> is a 
<p>-power-poly-pcp-group (thus representing an infinite coclass sequence 
through a pp-presentation) with generators $g_1, \ldots, g_n, t_1, 
\ldots, t_d, c_1, \ldots, c_m$ and <word> is a list of tuples, where the first 
entry <i> in the tuple gives the index of the generator (if $i \le n$, then 
it represents generator $g_i$, if $n \< i \le d$, then it represents generator
$t_{i-n}$ and otherwise it represents generator $c_{i-n-d}$) and the second 
entry of the tuple is the corresponding exponent. Note that the exponents 
of the $g_i$'s must be integers, while all other exponents can be integers 
or lists, representing an element depending on the parameter.

\beginexample
gap> G := PPPPcpGroups( ParPresGlobalVar_2_1[3] );
< P-Power-Poly-pcp-groups with 3 generators of relative orders [ 2,2,2*2^x ] >
gap> g1 := PPPPcpGroupsElement( G , [[1,1]] );
g1
gap> g := PPPPcpGroupsElement( G , [[1,1],[2,1],[3,1]] );
g1*g2*t1
gap> h := PPPPcpGroupsElement( G , [[1,1],[2,1],[3,G!.expo-1]] );
g1*g2*t1^(-1+2*2^x)
\endexample

%%%%%%%%%%%%%%%%%%%%%%%%%%%%%%%%%%%%%%%%%%%%%%%%%%%%%%%%%%%%%%%%%%%%%%%%%%%%%%%
\Section{Operations and functions for p-power-poly-pcp-group elements}

The typical operations for group elements can be carried out for 
<p>-power-poly-pcp-group elements, like `*', `/', Inverse, One, equality and 
ShallowCopy.

\>CollectPPPPcp( <a> ) F

collects the <p>-power-poly-pcp-group element <a> so that after reducing to
integers for every specific value for the parameter <x>, the element is 
collected in the polycyclic group, represented by <x> in the underlying 
pp-presentation. 

Note that the global
variable `COLLECT_PPOWERPOLY_PCP' determines whether every element will be 
collected immediately, when created, or not, see
%display{tex}
{\tt COLLECT_PPOWERPOLY_PCP},
%enddisplay
"COLLECT_PPOWERPOLY_PCP".

%%%%%%%%%%%%%%%%%%%%%%%%%%%%%%%%%%%%%%%%%%%%%%%%%%%%%%%%%%%%%%%%%%%%%%%%%%%%%%%
\Section{Operations and functions for p-power-poly-pcp-groups}

For <p>-power-poly-pcp-groups:

\> GeneratorsOfGroup( <G> )

returns a set of generators for the <p>-power-poly-pcp-groups <G>. 

\> One( <G> )

obtains the identity element of the <p>-power-poly-pcp-groups <G>.

\>IsConsistentPPPPcp( <G> ) F
\>IsConsistentPPPPcp( <ParPres> ) F

checks if the underlying pp-presentation of the 
<p>-power-poly-pcp-groups <G> is consistent or if the pp-presenta-tion 
<ParPres> is consistent.

\>GetPcGroupPPowerPoly( <ParPres>, <n> ) F
\>GetPcGroupPPowerPoly( <G>, <n> ) F

takes the pp-presentation given by the record <ParPres> as in
%display{tex}
{\tt PPPPcpGroups},
%enddisplay
"PPPPcpGroups" or the <p>-power-poly-pcp-groups <G> and takes <n>, a 
non-negative integer, as a value for the parameter to obtain a 
pc-presentation for the corresponding finite <p>-group.

\>GetPcpGroupPPowerPoly( <ParPres>, <n> ) F
\>GetPcpGroupPPowerPoly( <G>, <n> ) F

takes pp-presentation given by the record <ParPres> as in
%display{tex}
{\tt PPPPcpGroups},
%enddisplay
"PPPPcpGroups" or the <p>-power-poly-pcp-groups <G> and takes <n>, a 
non-negative integer, as the parameter to obtain a pcp-presentation for the 
corresponding finite <p>-group, for further information we refer to the 
polycyclic package.

\>GAPInputPPPPcpGroups( <file>, <G> ) F
\>GAPInputPPPPcpGroups( <file>, <ParPres> ) F

prints the <p>-power-poly-pcp-groups <G> defined by <ParPres> in the file 
<file> as a record that could be used as input to
%display{tex}
{\tt PPPPcpGroups},
%enddisplay
"PPPPcpGroups" to create <p>-power-poly-pcp-groups.

\>GAPInputPPPPcpGroupsAppend( <file>, <G> ) F
\>GAPInputPPPPcpGroupsAppend( <file>, <ParPres> ) F

appends the pp-presentation of the <p>-power-poly-pcp-groups <G> defined by 
<ParPres> to the file <file>  as a record that could be used as input to
%display{tex}
{\tt PPPPcpGroups},
%enddisplay
"PPPPcpGroups" to create <p>-power-poly-pcp-groups.

\>LatexInputPPPPcpGroups( <file>, <G> ) F
\>LatexInputPPPPcpGroups( <file>, <ParPres> ) F

prints the pp-presentation of <G> as given by <ParPres> in latex-code to the 
file <file>. Note that only non-trivial relations are printed.

\>LatexInputPPPPcpGroupsAppend( <file>, <G> ) F
\>LatexInputPPPPcpGroupsAppend( <file>, <ParPres> ) F

appends the pp-presentation of <G> as given by <ParPres> in latex-code to the 
file <file>. Note that only non-trivial relations are appended.

\> LatexInputPPPPcpGroupsAllAppend( <file>, <G> ) F
\> LatexInputPPPPcpGroupsAllAppend( <file>, <ParPres> ) F

appends the pp-presentation of <G> as given by <ParPres> in latex-code to the 
file <file>. Note that all relations are appended.

%%%%%%%%%%%%%%%%%%%%%%%%%%%%%%%%%%%%%%%%%%%%%%%%%%%%%%%%%%%%%%%%%%%%%%%%%%%%%%%
\Section{Info classes for the p-power-poly-pcp-groups}

The following info classes are available:

\>`InfoConsistencyPPPPcp' V

is an InfoClass with the following levels.

%display{nonhtml}
\beginitems
`level 1' & displays the first consistency relation that fails during the consistency check;

`level 2' & displays which family of consistency relations have been checked during a consistency check.
\enditems
%display{nontext}
%\beginitems
%`level 1' & displays the first consistency relation that fails during the consistency check;
%`level 2' & displays which family of consistency relations have been checked during a consistency check.
%\enditems
%enddisplay

the default value is 1.

\>`InfoCollectingPPPPcp' V

is an InfoClass with the following levels.

\beginitems
`level 1' & displays some information during collecting;
\enditems

the default value is 0.

%%%%%%%%%%%%%%%%%%%%%%%%%%%%%%%%%%%%%%%%%%%%%%%%%%%%%%%%%%%%%%%%%%%%%%%%%%%%%%%
\Section{Global variables for the p-power-poly-pcp-groups}

The following global variables are available with default value:

\>`COLLECT_PPOWERPOLY_PCP' V

is a global variable determining if every <p>-power-poly-pcp-group
element is collected, when created, the default value is true.
