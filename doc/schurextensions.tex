%%%%%%%%%%%%%%%%%%%%%%%%%%%%%%%%%%%%%%%%%%%%%%%%%%%%%%%%%%%%%%%%%%%%%%%%%
%%
%W  schurextension.tex     GAP documentation      D�rte Feichtenschlager
%%
%H  $Id: schurextension.tex, v 0.5 2010/05/31 09:30:00 gap SymbCompCC $
%%

%%%%%%%%%%%%%%%%%%%%%%%%%%%%%%%%%%%%%%%%%%%%%%%%%%%%%%%%%%%%%%%%%%%%%%%%%
\Chapter{Schur extensions for p-power-poly-pcp-groups}

In this chapter we describe how the consistent pp-presentations
of infinite coclass sequences can be used to compute a pp-presentation for 
the corresponding Schur extensions (see \cite{EF11}).

For a group $G = F/R$ the Schur extension $H$ is defined as $H = F/[F,R]$ 
(see \cite{EN08}).

So for a parameter <x> that can take values in the positive integers, let 
$(G_x = F/R_x | x \in \N)$, for $\N$ the positive integers, describe an 
infinite coclass sequence of finite $p$-groups $G_X$ of coclass $r$. Then for 
each value for the parameter <x>, the group $G_x$ has a consistent polycyclic 
presentation with generators $g_1, ..., g_n, t_1, ..., t_d$ and relations

%display{nontext}
$$
\eqalign{
&\, g_i^p = rel[i][i],\cr
&\, t_i^{expo} = rel[n+i][n+i],\cr
&\, g_i^{g_j} = rel[j][i],\cr
&\, t_i^{g_j} = rel[j][n+i],\cr
&\, t_i^{t_j} = 1.
}
$$
%display{text}
%g_i^p = rel[i][i],
%t_i^{expo} = rel[n+i][n+i],
%g_i^{g_j} = rel[j][i],
%t_i^{g_j} = rel[j][n+i],
%t_i^{t_j} = 1.
%enddisplay

Then we compute a consistent pp-presentation of the corresponding Schur 
extensions of with generators $g_1, ..., g_n, t_1, ..., t_d, c_1, ... c_m$ and
relations

%display{nontext}
$$
\eqalign{
&\, g_i^p=rel[i][i],\cr
&\, t_i^{expo} = rel[n+i][n+i],\cr
&\, c_i^{expo\_vec[i]} = rel[n+d+i,n+d+i],\cr
&\, g_i^{g_j} = rel[j][i], \cr
&\, t_i^{g_j} = rel[j][n+i],\cr
&\, t_i^{t_j} = rel[n+j][n+i],\cr
&\, c_i^{g_j} = 1, \cr
&\, c_i^{t_j} = 1, \cr
&\, c_i^{c_j} = 1.
}
$$
%display{text}
%g_i^p=rel[i][i],
%t_i^{expo}=rel[n+i][n+i],
%c_i^{expo\_vec[i]}=rel[n+d+i,n+d+i],
%g_i^{g_j} = rel[j][i], 
%t_i^{g_j} = rel[j][n+i],
%t_i^{t_j} = rel[n+j][n+i],
%c_i^{g_j} = 1, 
%c_i^{t_j} = 1, 
%c_i^{c_j} = 1.
%enddisplay

where the $t_i$'s commute modulo $< c_1, ..., c_m>$ and the $c_i$'s are 
central. 

%%%%%%%%%%%%%%%%%%%%%%%%%%%%%%%%%%%%%%%%%%%%%%%%%%%%%%%%%%%%%%%%%%%%%%%%%
\Section{Computing Schur extensions}

\>SchurExtParPres( <G> ) 

computes the Schur extensions corresponding to the <p>-power-poly-pcp-groups
<G> and returns them as <p>-power-poly-pcp-groups.

\>SchurExtParPres( <ParPres> ) F

computes a consistent pp-presentation of Schur extensions of the 
groups defined by the record <ParPres> which describes 
<p>-power-poly-pcp-groups. The output is a record 
<rec>(<rel>, <expo>, <n>, <d>, <m>, <prime>, <cc>, <expo\_vec>, <name>), 
which describes the Schur extensions as <p>-power-poly-pcp-groups; it is 
encoded in a form that it can be used as input for 
%display{tex}
{\tt PPPPcpGroups},
%enddisplay
"PPPPcpGroups".

\beginexample
gap> SchurExtParPres( ParPresGlobalVar_2_1[1] );
rec( prime := 2, 
  rel := [ [ [ [ 7, 1 ] ] ], [ [ [ 2, 1 ], [ 3, -1+2*2^x ], [ 6, 1-2*2^x ] ], 
          [ [ 3, 1 ], [ 5, 1 ] ] ], 
      [ [ [ 3, -1+2*2^x ], [ 4, 1 ], [ 6, 2-2*2^x ] ], [ [ 3, 1 ] ], 
          [ [ 4, 1 ], [ 6, 2*2^x ] ] ], 
      [ [ [ 4, 1 ] ], [ [ 4, 1 ] ], [ [ 4, 1 ] ], [ [ 4, 0 ] ] ], 
      [ [ [ 5, 1 ] ], [ [ 5, 1 ] ], [ [ 5, 1 ] ], [ [ 5, 1 ] ], [ [ 5, 0 ] ] ]
        , 
      [ [ [ 6, 1 ] ], [ [ 6, 1 ] ], [ [ 6, 1 ] ], [ [ 6, 1 ] ], [ [ 6, 1 ] ], 
          [ [ 6, 0 ] ] ], 
      [ [ [ 7, 1 ] ], [ [ 7, 1 ] ], [ [ 7, 1 ] ], [ [ 7, 1 ] ], [ [ 7, 1 ] ], 
          [ [ 7, 1 ] ], [ [ 7, 0 ] ] ] ], n := 2, d := 1, m := 4, 
  expo := 2*2^x, expo_vec := [ 2, 0, 0, 0 ], cc := fail, name := "SchurExt_D" 
 )
\endexample

%%%%%%%%%%%%%%%%%%%%%%%%%%%%%%%%%%%%%%%%%%%%%%%%%%%%%%%%%%%%%%%%%%%%%%%%%
\Section{Computing other invariants from Schur extensions}

\>SchurMultiplicatorsStructurePPPPcps( <G> ) F

computes the abalian invariants of the Schur multiplicators <M(G)> of the 
<p>-power-poly-pcp-groups <G>. The output is a list $[d_1, ..., d_k]$ 
consisting elements $d_i$, depending on the underlying parameter, such that 
$M(G) \cong C_{d_1} \times \ldots \times C_{d_k}$.

\beginexample
gap> G := PPPPcpGroups( ParPresGlobalVar_2_1[1] );
< P-Power-Poly pcp-groups with 3 generators of relative orders [ 2,2,2*2^x ] >
SchurMultiplicatorsStructurePPPPcps( G );
[ 2 ]
\endexample

\>SchurMultiplicator( <G> )!{for p-power-poly-pcp-groups} F

computes the Schur multiplicators of the <p>-power-poly-pcp-groups <G> and 
then returns the corresponding 
%display{tex}
{\tt PPPPcpGroups},
%enddisplay
"PPPPcpGroups".

\beginexample
gap> G := PPPPcpGroup( ParPresGlobalVar_3_1[1] );
< P-Power-Poly pcp-group with 5 generators of relative orders [ 3,3,3,3*3^x,3*3^x ] >
gap> SchurMultiplicator( G );
< P-Power-Poly pcp-groups with 2 generators of relative orders [ 3,9*3^x ] >
\endexample

\>AbelianInvariants( <G> )!{for p-power-poly-pcp-groups} F

computes the abelian invariants of the <p>-power-poly-pcp-groups <G> and returns
them as a list of list describing the parametrised elements.

\beginexample
gap> G := PPPPcpGroups( ParPresGlobalVar_2_1[1] );
< P-Power-Poly pcp-groups with 3 generators of relative orders [ 2,2,2*2^x ] >
gap> AbelianInvariants( G );
[ 2, 2 ]
\endexample

\>ZeroCohomologyPPPPcps( <G>[, <p>] ) F

computes the zero-th-cohomology groups $H^0(G,R)$ of the 
<p>-power-poly-pcp-groups <G> with coefficients in $R$, where $R \cong GF(p)$ if
the prime $p$ is given or $R \cong \Z$ otherwise. The action of $G$ on $R$ is
taken to be trivial. The function returns a list of integers $[a_1,\ldots,
a_k]$ where the cohomology group is isomorphic to $C_{a_1} \times \ldots 
\times C_{a_k}$ with $C_i$ a cyclic group of order $i$ (for $i > 0$) and $C_0$ 
is interpreted as $\Z$.

\beginexample
gap> G := PPPPcpGroups( ParPresGlobalVar_2_1[1] );
< P-Power-Poly pcp-groups with 3 generators of relative orders [ 2,2,2*2^x ] >
gap> ZeroCohomologyPPPPcp( G, 2 );
[ 2 ]
\endexample

\>FirstCohomologyPPPPcps( <G>[, <p>] ) F

computes the first-cohomology groups $H^1(G,R)$ of the 
<p>-power-poly-pcp-groups <G> with coefficients in $R$, where $R \cong GF(p)$ if
the prime $p$ is given or $R \cong \Z$ otherwise. The action of $G$ on $R$ is
taken to be trivial. The function returns a list of integers $[a_1,\ldots,
a_k]$ where the cohomology group is isomorphic to $C_{a_1} \times \ldots 
\times C_{a_k}$ with $C_i$ a cyclic group of order $i$ (for $i > 0$) and $C_0$ 
is interpreted as $\Z$.

\beginexample
gap> G := PPPPcpGroups( ParPresGlobalVar_2_1[1] );
< P-Power-Poly pcp-groups with 3 generators of relative orders [ 2,2,2*2^x ] >
gap> FirstCohomologyPPPPcps( G );
[  ]
\endexample

\>SecondCohomologyPPPPcps( <G>[, <p>] ) F

computes the second-cohomology groups $H^2(G,R)$ of the 
<p>-power-poly-pcp-groups <G> with coefficients in $R$, where $R \cong GF(p)$ if
the prime $p$ is given or $R \cong \Z$ otherwise. The action of $G$ on $R$ is
taken to be trivial. The function returns a list of integers $[a_1,\ldots,
a_k]$ where the cohomology group is isomorphic to $C_{a_1} \times \ldots 
\times C_{a_k}$ with $C_i$ a cyclic group of order $i$ (for $i > 0$) and $C_0$ 
is interpreted as $\Z$.

\beginexample
gap> G := PPPPcpGroups( ParPresGlobalVar_2_1[1] );
< P-Power-Poly pcp-groups with 3 generators of relative orders [ 2,2,2*2^x ] >
gap> SecondCohomologyPPPPcps( G, 2 );
[ 2, 2, 2 ]
\endexample

%%%%%%%%%%%%%%%%%%%%%%%%%%%%%%%%%%%%%%%%%%%%%%%%%%%%%%%%%%%%%%%%%%%%%%%%%
\Section{Info classes for the computation of the Schur extension}

The following info classes are available

\>`InfoConsistencyRelPPowerPoly' V

\beginitems
`level 1' & shows which consistency relations are computed and gives the 
result;
\enditems

the default value is 0.

\>`InfoCollectingPPowerPoly' V

\beginitems
`level 1' & shows what is done during collecting;
\enditems

the default value is 0.
