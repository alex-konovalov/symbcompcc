%%%%%%%%%%%%%%%%%%%%%%%%%%%%%%%%%%%%%%%%%%%%%%%%%%%%%%%%%%%%%%%%%%%%%%%%%
%%
%W  install.tex            GAP documentation      D�rte Feichtenschlager
%%
%H  $Id: install.tex, v0.5 2010/05/31 09:30:00 gap SymbCompCC $
%%

%%%%%%%%%%%%%%%%%%%%%%%%%%%%%%%%%%%%%%%%%%%%%%%%%%%%%%%%%%%%%%%%%%%%%%%%%
\Chapter{Installing and Loading the SymbCompCC Package}

%%%%%%%%%%%%%%%%%%%%%%%%%%%%%%%%%%%%%%%%%%%%%%%%%%%%%%%%%%%%%%%%%%%%%%%%%
\Section{Installing the SymbCompCC Package}

The following installation instruction is for unix although the package
should work as well with any other operating system.

To install the {\SymbCompCC} package, unpack the archive file, which  should
have a name of the form `SymbCompCC-<XXX>.tar.bz2' for some version number 
<XXX>, by typing

\){\kernttindent}bunzip2 SymbCompCC-<XXX>.tar.bz2
\){\kernttindent}tar -xvf SymbCompCC-<XXX>.tar

in the `pkg' directory of your version of {\GAP}~4,  or  in  a  directory
named `pkg' (e.g.~in your home directory). (The only essential difference
with installing {\SymbCompCC} in a `pkg' directory different to the {\GAP}~4
home directory is that one  must  start  {\GAP}  with  the  `-l'  switch,
e.g.~if your private `pkg' directory is a subdirectory of `mygap' in your
home directory you might type:

%begintt
\){\kernttindent}gap -l ";<myhomedir>/mygap"
%endtt

where <myhomedir> is the  path  to  your  home  directory,  which  (since
{\GAP}~4.3) may be replaced  by  a  tilde.  The  empty  path  before  the
semicolon is  filled  in  by  the  default  path  of  the  {\GAP}~4  home
directory.)

%%%%%%%%%%%%%%%%%%%%%%%%%%%%%%%%%%%%%%%%%%%%%%%%%%%%%%%%%%%%%%%%%%%%%%%%%
\Section{Loading the SymbCompCC Package}

To use the {\SymbCompCC} Package you have to request it explicitly. This  is
done by calling

\beginexample
gap> LoadPackage("SymbCompCC");
true
\endexample

The `LoadPackage' command is described  in  Section~"ref:LoadPackage"  in
the {\GAP} Reference Manual.

If you want to load the {\SymbCompCC} package by default, you  can  put  the
`LoadPackage' command  into  your  `.gaprc'  file  (see  Section~"ref:The
.gaprc file" in the {\GAP} Reference Manual).

%%%%%%%%%%%%%%%%%%%%%%%%%%%%%%%%%%%%%%%%%%%%%%%%%%%%%%%%%%%%%%%%%%%%%%%%%
%%
%E
